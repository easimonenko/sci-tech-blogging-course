% slides-topic-and-preparation.tex
% How to choose a topic and prepare for writing a post.
% Language: Russian
% Author: Evgeny Simonenko <easimonenko@mail.ru>
% License: CC BY-ND 4.0

\documentclass[12pt]{beamer}

\usepackage{polyglossia}
\setdefaultlanguage{russian}
\setotherlanguage{english}
\defaultfontfeatures{Ligatures={TeX},Renderer=Basic}
\setmainfont{FreeSerif}
\setsansfont{FreeSans}
\setmonofont[SizeFeatures={Size=10}]{FreeMono}

\title[Блогерство]{Научно-технологическое блогерство}
\subtitle{Как найти тему для поста и подготовиться к его написанию}
\author[]{Симоненко Евгений}
\institute[]{Университет ИТМО}
\date[]{Санкт-Петербург, 2022}

\begin{document}

\begin{frame}
  \titlepage
\end{frame}

\begin{frame}
  \frametitle{Содержание}
  \tableofcontents
\end{frame}

\section{Как найти тему для поста}

\begin{frame}
  \frametitle{Источники вдохновения для темы}
  Вопрос: если вы уже писали посты, то откуда вы черпали вдохновение для темы?
  А если не писали, то умозрительно, что бы это могло быть?
\end{frame}

\begin{frame}
  \frametitle{Источники вдохновения для темы}
  \begin{itemize}
  \item накопившийся опыт в какой-то области
  \item встреча с интересной новой технологией или открытием
  \item впечатления от путешествия, посещения выставки, музея, лаборатории
  \item прочитанный пост и желание дополнить или ответить
  \item важное событие в вашей жизни, которое можно экстраполировать
  \item хайповая тема, на которую вам тоже есть что сказать
  \end{itemize}
\end{frame}

\section{Как подготовиться к написанию поста}

\begin{frame}
  \frametitle{Подготовка к написанию поста}
  \begin{enumerate}
  \item Определитесь со своей мотивацией.
  \item Определитесь с целью написания поста.
  \item Сформулируйте и выпишите заголовок поста.
  \item Выясните, не писал ли кто уже об этом.
  \item Решите, чем ваш пост будет отличаться от похожих.
  \item Определитесь с жанром поста.
  \item Соберите дополнительную информацию, изображения, ссылки.
  \item Составьте содержание (оглавление) поста.
  \item Попробуйте рассказать будущий текст самому себе.
  \item Напишите первоначальный вариант лида поста.
  \end{enumerate}
\end{frame}

\begin{frame}
  \frametitle{Подготовка к написанию поста}
  Что не следует делать при подготовке:
  \begin{enumerate}
  \item Ждать вдохновения извне и, в принципе, его ждать.
  \item Сомневаться в своих способностях.
  \item Заранее переживать, что пост не зайдёт.
  \item Затягивать, откладывать подготовку и начало работы над постом.
  \item Собирать все возможные материалы и ссылки.
  \item Садиться за компьютер, не зная точно, что будете делать.
  \item Отвлекаться, вырывать себя из потока.
  \item Доводить себя до изнеможения (переутомляться, не спать по ночам).
  \end{enumerate}
\end{frame}

\section*{Благодарность}

\begin{frame}
  \center
  \textit{Благодарю за внимание!}

  \textbf{\textsl{\inserttitle}}

  \insertauthor

  \url{mailto:easimonenko@mail.ru}

  \insertinstitute
\end{frame}

\end{document}
