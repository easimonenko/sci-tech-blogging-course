% slides-authors-errors.tex
% What errors do blog authors make?
% Language: Russian
% Author: Evgeny Simonenko <easimonenko@mail.ru>
% License: CC BY-ND 4.0

\documentclass[12pt]{beamer}

\usepackage{polyglossia}
\setdefaultlanguage{russian}
\setotherlanguage{english}
\defaultfontfeatures{Ligatures={TeX},Renderer=Basic}
\setmainfont{FreeSerif}
\setsansfont{FreeSans}
\setmonofont[SizeFeatures={Size=10}]{FreeMono}

\title[Блогерство]{Научно-технологическое блогерство}
\subtitle{Какие ошибки допускают авторы блогов?}
\author[]{Симоненко Евгений}
\institute[]{Университет ИТМО}
\date[]{Санкт-Петербург, 2022}

\begin{document}

\begin{frame}
  \titlepage
\end{frame}

\begin{frame}
  \frametitle{Содержание}
  \tableofcontents
\end{frame}

\section{Ошибки в блогах}

\begin{frame}
  \frametitle{Ошибки в блогах}
  \begin{itemize}
  \item Отсутствие структуры (тегов, категорий, разделов).
  \item Проблемы с юзабилити (цвета, шрифты, интерфейс).
  \item Отсутствие позиционирования (к вопросу о мотивации и целях автора / владельца).
  \end{itemize}
\end{frame}

\section{Общие ошибки в постах}

\begin{frame}
  \frametitle{Общие ошибки в постах}
  \begin{itemize}
  \item Нечестность.
  \item Фамильярность, заигрывание и сюсюкание с читателем.
  \item Грамматические ошибки, опечатки.
  \item Англицизмы и непереведённые на русский язык слова.
  \item Непривлекательный и неинформативный заголовок поста.
  \item Отсутствие лида (картикой его очень сложно заменить).
  \item Отсутствие структуры основной части.
  \item Резкое обрывание повествования.
  \end{itemize}
\end{frame}

\section{Ошибки в заголовках}

\begin{frame}
  \frametitle{Ошибки в заголовках}
  \begin{itemize}
  \item Слишком короткий и неинформативный заголовок. Из такого неясно,
    о чём будет пост.
  \item Слишком длинный и сложносочинённый заголовок. Такой трудно читать.
  \item Обозначение слишком общей темы.
  \item ТЕКСТ КАПСОМ (заглавными буквами, CAPS LOCK).
  \item Слова с Большой Буквы (Английский Стиль).
  \item Точка в конце заголовка.
  \end{itemize}
\end{frame}

\begin{frame}
  \frametitle{Ошибки в заголовках}
  Примеры плохих и хороших заголовков
  \begin{itemize}
  \item О производстве хлебо-булочных изделий
  \item Как устроен хлеб-завод, и куда девается непроданный хлеб
  \end{itemize}
  \begin{itemize}
  \item Популярные языки программирования
  \item Как измеряется популярность языков программирования
  \end{itemize}
\end{frame}

\section{Ошибки в лиде}

\begin{frame}
  \frametitle{Ошибки в лиде}
  \begin{itemize}
  \item Отсутствие лида.
  \item Подмена лида вводными словами.
  \item Картинка не по теме (просто чтобы была).
\end{itemize}
\end{frame}

\section{Ошибки в основной части}

\begin{frame}
  \frametitle{Ошибки в основной части}
  \begin{itemize}
  \item Словоблудие (отсебятина, множество неподкреплённых фактами слов, утверждений).
  \item Непонимание излагаемого материала.
  \item Отсутствие личного опыта с тем, о чём пишете.
  \item Злоупотребление картинками не по теме (мемасики).
  \item Отсутствие иллюстраций или примеров кода при объяснении сложного материала.
  \end{itemize}
\end{frame}

\section{Ошибки в заключительных словах}

\begin{frame}
  \frametitle{Ошибки в заключительных словах}
  \begin{itemize}
  \item Отсутствие заключительных слов (вызывает разочарование, неудовлетворённость у читателей).
  \item Обозначение заголовком `Заключение`.
  \item Краткое изложение материала поста.
  \item Просьба быть к автору снисходительным.
  \item Выражение озабоченности тем, понравился ли читателю ваш пост.
  \end{itemize}
\end{frame}

\section{Ссылки}

\begin{frame}
  \frametitle{Ссылки}
  \begin{itemize}
  \item \url{https://ru.wikipedia.org/wiki/Кликбейт}
  \end{itemize}
\end{frame}

\section*{Благодарность}

\begin{frame}
  \center
  \textit{Благодарю за внимание!}

  \textbf{\textsl{\inserttitle}}

  \insertauthor

  \url{mailto:easimonenko@mail.ru}

  \insertinstitute
\end{frame}

\end{document}
