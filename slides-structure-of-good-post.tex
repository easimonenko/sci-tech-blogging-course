% slides-structure-of-good-post.tex
% How can a good post be structured?
% Language: Russian
% Author: Evgeny Simonenko <easimonenko@mail.ru>
% License: CC BY-ND 4.0

\documentclass[12pt]{beamer}

\usepackage{polyglossia}
\setdefaultlanguage{russian}
\setotherlanguage{english}
\defaultfontfeatures{Ligatures={TeX},Renderer=Basic}
\setmainfont{FreeSerif}
\setsansfont{FreeSans}
\setmonofont[SizeFeatures={Size=10}]{FreeMono}

\title[Блогерство]{Научно-технологическое блогерство}
\subtitle{Как можно структурировать хороший пост?}
\author[]{Симоненко Евгений}
\institute[]{Университет ИТМО}
\date[]{Санкт-Петербург, 2022}

\begin{document}

\begin{frame}
  \titlepage
\end{frame}

\begin{frame}
  \frametitle{Содержание}
  \tableofcontents
\end{frame}

\section{Общая структура поста}

\begin{frame}
  \frametitle{Общая структура поста}
  Структура поста зависит от его жанра, но всё же...

  Типичный пост включает в себя следующие части:
  \begin{itemize}
  \item Заголовок поста.
  \item Информация об авторстве, дата публикации, теги, категория.
  \item Лид (вводный абзац).
  \item Пререквизиты.
  \item Вводные слова.
  \item Основной текст.
  \item Заключительные слова.
  \item Ссылки, источники, дополнительные материалы.
  \item Опрос*.
  \end{itemize}
\end{frame}

\section{Заголовок поста}

\begin{frame}
  \frametitle{Заголовок поста}
  \begin{itemize}
  \item Заголовок -- первое, что видит читатель.
  \item И от него зависит, будет ли читатель читать дальше.
  \item Поэтому он должен быть информативным и честным.
  \item А не только привлекательным (clickbait).
  \item Может иметь подзаголовок.
  \item Задача подзаголовка -- уточнить основную тему.
  \end{itemize}
\end{frame}

\begin{frame}
  \frametitle{Заголовок поста}
  Заголовок не просто отражает тему поста. Он содержит в себе:
  \begin{itemize}
  \item предмет обсуждения
  \item результат обсуждения
  \end{itemize}
  Например: Почему домашние пирожки вкуснее магазинных, и как нам удалось испечь
  их такими же в нашей пекарне.
\end{frame}

\section{Лид поста}

\begin{frame}
  \frametitle{Лид поста}
  Лид поста -- начальный вводный абзац поста.
  \begin{itemize}
  \item его задача -- рассказать о чём будет пост
  \item и сделать это так, чтобы пост захотелось прочитать
  \item он обязателен
  \item вводные слова его не заменяют
  \item выделяется визуально, как бы отдельно от остального текста
  \item \emph{может} включать в себя привлекательную картинку
  \item написать хороший лид очень сложно
  \end{itemize}
\end{frame}

\section{Пререквизиты}

\begin{frame}
  \frametitle{Пререквизиты}
  Пререквизиты -- требования к читателю и ограничения на материал.
  \begin{itemize}
  \item Иногда нужно заранее указать, какие знания у читателя уже должны быть,
    чтобы он смог разобраться в том, о чем ваш пост.
  \item Также автор может проинформировать читателя о допущениях при написании
    им этого текста.
  \item Чаще всего встречаются в туториалах.
  \end{itemize}
\end{frame}

\section{Вводные слова}

\begin{frame}
  \frametitle{Вводные слова}
  Вводные слова -- один или несколько абзацев текста свободного содержания.
  \begin{itemize}
  \item Это может быть какая-то история автора.
  \item Это может быть предыстория вопроса.
  \item Лид и вводные слова -- разные вещи.
  \item Вводные слова не тоже самое, что введение в научных статьях.
  \item Идут сразу за лидом и структурно не выделяются (без заголовка).
  \end{itemize}
\end{frame}

\section{Основной текст}

\begin{frame}
  \frametitle{Основной текст}
  Основной текст -- собственно основное содержимое поста, то, ради чего пост читают.
  \begin{itemize}
  \item Короткая основная часть может не иметь структуры, но длинный пост -- обязан.
  \item Основной текст должен раскрыть заголовок поста и соответствовать его лиду.
  \item Хороший вариант: писать в стиле истории.
  \item Пишите от первого лица (в тексте должен чувствоваться автор).
  \item Используйте иллюстрации, примеры программного кода, таблицы.
  \item Врезки и спойлеры.
  \end{itemize}
\end{frame}

\section{Заключительные слова}

\begin{frame}
  \frametitle{Заключительные слова}
  Заключительные слова завершают текст поста.
  \begin{itemize}
  \item Не путайте с заключением в научных статьях.
  \item Призваны создать ощущение завершённости текста поста.
  \item По содержанию могут быть о чём угодно.
  \item Например: пообещать рассказать в следующем посте о чём-то более подробно.
  \item Например: подвести итог, суммировать.
  \item Например: порекомендовать дополнительные материалы.
  \end{itemize}
\end{frame}

\section{Опрос}

\begin{frame}
  \frametitle{Опрос}
  Некоторые площадки предоставляют возможность разместить в посте опрос для читателей.
  \begin{itemize}
  \item Используйте только, если вам действительно нужно собрать какую-то информацию.
  \item Результат опроса должен быть интересен и читателю.
  \item Не злоупотребляйте (не помещайте в каждый свой пост, и не более двух-трёх за раз).
  \item Избегайте опроса ради опроса.
  \item Подготовить качественный опрос очень сложно.
  \end{itemize}
\end{frame}

\section{Ссылки}

\begin{frame}
  \frametitle{Ссылки}
  \begin{itemize}
  \item Пишем на Хабр: просто о технически сложном (видео от Хабра):
    \url{https://www.youtube.com/watch?v=L87caoiUPS0}
  \end{itemize}
\end{frame}

\section*{Благодарность}

\begin{frame}
  \center
  \textit{Благодарю за внимание!}

  \textbf{\textsl{\inserttitle}}

  \insertauthor

  \url{mailto:easimonenko@mail.ru}

  \insertinstitute
\end{frame}

\end{document}
