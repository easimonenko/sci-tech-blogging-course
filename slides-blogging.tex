% slides-blogging.tex
% Blogging and blog publishing platforms
% Language: Russian
% Author: Evgeny Simonenko <easimonenko@mail.ru>
% License: CC BY-ND 4.0

\documentclass[14pt]{beamer}

\usepackage{polyglossia}
\setdefaultlanguage{russian}
\setotherlanguage{english}
\defaultfontfeatures{Ligatures={TeX},Renderer=Basic}
\setmainfont{FreeSerif}
\setsansfont{FreeSans}
\setmonofont[SizeFeatures={Size=10}]{FreeMono}

\title[Блогерство]{Научно-технологическое блогерство}
\subtitle{Блогерство и площадки для публикации блогов}
\author[]{Симоненко Евгений}
\institute[]{Университет ИТМО}
\date[]{Санкт-Петербург, 2022}

\begin{document}

\begin{frame}
  \titlepage
\end{frame}

\begin{frame}
  \frametitle{Содержание}
  \tableofcontents
\end{frame}

\section{Феномен блогерства}

\begin{frame}
  \frametitle{Блоги и блогерство}
  web log -> blog

  \emph{Блог} -- небольшой сайт или раздел на сайте, несущий авторский
  характер и содержащий небольшие материалы, посвящённые одной конкретной
  вещи.

  Чаще всего отдельные материалы блога называют \emph{постами} (от англ. слова \textit{post}).
\end{frame}

\begin{frame}
  \frametitle{Блоги и блогерство}
  \emph{Блогерство} -- деятельность по созданию небольших публикаций,
  отражающих авторский взгляд на вещи.

  Это могут быть:
  \begin{itemize}
  \item уникальный опыт автора
  \item мнение автора
  \item событие в жизни автора или сообщества
  \end{itemize}
\end{frame}

\begin{frame}
  \frametitle{Блог vs СМИ}
  Различия между блогами и блогерами и средствами массовой информации:
  \begin{itemize}
  \item Блогер не обязан регистрироваться как СМИ
  \item Блогер не является журналистом, но может им быть
  \item Блогер может работать в СМИ
  \item Компании (организации) могут вести свой блог
  \end{itemize}
\end{frame}

\section{Мотивация и цели авторов и их читателей}

\begin{frame}
  \frametitle{Мотивация авторов}
  Вопрос: как вы думаете, какая у авторов бывает мотивация писать посты или вести свой блог?
\end{frame}

\begin{frame}
  \frametitle{Мотивация авторов}
  Возможные варианты (открытый список):
  \begin{itemize}
  \item желание делиться опытом с другими
  \item желание высказаться, сообщить своё мнение
  \item желание быть причастным к науке, технологии, общественным движениям
  \item желание структурировать и сохранить личный опыт на будущее
  \item самореализация, публикация своего творчества
  \item желание сформировать личный или корпоративный бренд и имидж
  \item желание заработать деньги
  \item ... (вставьте своё)
  \end{itemize}
\end{frame}

\begin{frame}
  \frametitle{Цели авторов}
  Вопрос: как вы думаете, какие у авторов бывают цели писать посты или вести свой блог?
\end{frame}

\begin{frame}
  \frametitle{Цели авторов}
  Возможные варианты (открытый список):
  \begin{itemize}
  \item привлечь внимание к области науки, технологии, общественной проблеме
  \item инициировать создание сообщества вокруг области науки или технологии
  \item привлечь внимание потенциальных абитуриентов к образовательной программе
  \item опосредованная или прямая реклама компании и её продуктов
  \item получить обратную связь от сообщества / читателей
  \item ... (вставьте своё)
  \end{itemize}
\end{frame}

\begin{frame}
  \frametitle{Мотивация читателей}
  Вопрос: как вы думаете, какая у читателей бывает мотивация читать посты
  или подписываться на блоги?
\end{frame}

\begin{frame}
  \frametitle{Мотивация читателей}
  Возможные варианты (открытый список):
  \begin{itemize}
  \item желание развлечься, почитать / посмотреть / послушать на выходных / перед сном
  \item желание профессионального и интеллектуального развития
  \item желание быть причастным к науке, технологии, общественным движениям
  \item желание узнать что-то новое, неожиданное
  \item желание найти решение своей проблемы
  \item ... (вставьте своё)
  \end{itemize}
\end{frame}

\begin{frame}
  \frametitle{Цели читателей}
  Вопрос: как вы думаете, какие у читателей бывают цели читать посты или подписываться на блоги?
\end{frame}

\begin{frame}
  \frametitle{Цели читателей}
  Возможные варианты (открытый список):
  \begin{itemize}
  \item получить удовольствие от прочтения любимого автора
  \item решить сложную проблему
  \item получить информацию из первых рук
  \item узнать что-то, о чём в книгах ещё не написали / не напишут
  \item ... (вставьте своё)
  \end{itemize}
\end{frame}

\begin{frame}
  \frametitle{Мотивация и цели}
  Что ещё нужно знать про мотивацию и цели:
  \begin{itemize}
  \item Неплохо осознавать свою мотивацию как к ведению блога, так и к написанию постов.
  \item Важно при ведении блога, но особенно перед написанием постов, ставить конкретные цели.
  \item Мотивация и цели автора и читателя могут не совпадать.
  \item Вы как автор можете не достичь своих целей. Так бывает, и это реальность.
  \item Мотивация и цели автора и его читателей могут быть комплиментарны.
  \end{itemize}
\end{frame}

\section{Блог-площадки}

\begin{frame}
  \frametitle{Площадки общего характера}
  \begin{itemize}
  \item ЖЖ \url{https://www.livejournal.com/}
  \item Дзен \url{https://dzen.ru/}
  \item VK \url{https://vk.com/}
  \item Telegram \url{https://telegram.org/}
  \item Telegraph \url{https://telegra.ph/}
  \item Хабр \url{https://habr.com/}
  \end{itemize}
\end{frame}

\begin{frame}
  \frametitle{Хабр: он один такой}
  \begin{itemize}
  \item тематика разная, но доминирует о программировании, компьютерах, электронике
  \item много постов на тему менеджмента, образования, рынка труда
  \item пишут также о науке и других технологиях
  \item аудитория соответствующая
  \end{itemize}
\end{frame}

\begin{frame}
  \frametitle{Хабр: особенности публикации}
  \begin{itemize}
  \item посты помещаются в один или более хабов
  \item посты относятся к одному из видов (статья, туториал, новость)
  \item посты можно плюсовать / минусовать
  \item посты можно комментировать
  \item комментарии можно плюсовать / минусовать
  \item у авторов есть карма, которая также может плюсоваться / минусоваться
  \item лицензия на посты CC-BY-SA
  \end{itemize}
\end{frame}

\begin{frame}
  \frametitle{Хабр: как стать автором}
  Всё просто (ну, почти):
  \begin{enumerate}
  \item Зарегистрироваться на Хабре
  \item Написать пост с адекватной Хабру темой
  \item Опубликовать пост в песочнице Хабра
  \item Пройти неявную модерацию от редакции Хабра
  \item Получить инвайт от одного из авторов Хабра
  \item После инвайта пост окажется в общей ленте
  \item Надеяться, что пост зайдёт
  \item Теперь Вы автор Хабра!
  \end{enumerate}
\end{frame}

\begin{frame}
  \frametitle{Хабр: как перестать быть автором}
  Всё ещё проще:
  \begin{enumerate}
  \item Напишите низкосортный / откровенно рекламный / высокомерный пост.
  \item Пишите такие же комментарии.
  \item Добивайтесь тем самым минусования вашей кармы.
  \item Карма в минусе: вы больше не автор. Profit!
  \end{enumerate}
\end{frame}

\section{Что почитать}

\begin{frame}
  \frametitle{Что почитать}
  \begin{itemize}
  \item Блог (Wikipedia): \url{https://ru.wikipedia.org/wiki/Блог}
  \item Как стать автором (Хабр): \url{https://habr.com/ru/sandbox/start/}
  \item Кодекс авторов Хабра: \url{https://habr.com/ru/docs/authors/codex/}
  \end{itemize}
\end{frame}

\section*{Благодарность}

\begin{frame}
  \center
  \textit{Благодарю за внимание!}

  \textbf{\textsl{\inserttitle}}

  \insertauthor

  \url{mailto:easimonenko@mail.ru}

  \insertinstitute
\end{frame}

\end{document}
