% slides-about-course.tex
% About course
% Language: Russian
% Author: Evgeny Simonenko <easimonenko@mail.ru>
% License: CC BY-ND 4.0

\documentclass[12pt]{beamer}

\usepackage{polyglossia}
\setdefaultlanguage{russian}
\setotherlanguage{english}
\defaultfontfeatures{Ligatures={TeX},Renderer=Basic}
\setmainfont{FreeSerif}
\setsansfont{FreeSans}
\setmonofont[SizeFeatures={Size=10}]{FreeMono}

\title[Блогерство]{Научно-технологическое блогерство}
\subtitle{О курсе}
\author[]{Симоненко Евгений}
\institute[]{Университет ИТМО}
\date[]{Санкт-Петербург, 2022}

\begin{document}

\begin{frame}
  \titlepage
\end{frame}

\begin{frame}
  \frametitle{Содержание}
  \tableofcontents
\end{frame}

\section{Об авторе курса}

\begin{frame}
  \frametitle{Об авторе курса}
  \begin{itemize}
  \item Изучает научную коммуникацию в Университете ИТМО
  \item По образованию математик
  \item По профессии программист и преподаватель
  \item Пишет на Хабре \url{https://habr.com/ru/users/easimonenko/posts/}
  \item Ведёт свой блог \url{https://easimonenko.github.io/}
  \end{itemize}
\end{frame}

\section{О курсе}

\begin{frame}
  \frametitle{О чём этот курс}
  \begin{itemize}
  \item Мотивация и цели авторов блогов и их читателей
  \item Площадки для публикации постов
  \item Жанры постов о науке и технологиях
  \item Как найти и сформулировать тему поста
  \item Как подготовиться к написанию поста
  \item Структура поста
  \item Ошибки авторов
  \item Markdown и Jekyll
  \end{itemize}
\end{frame}

\section{Как проходят занятия}

\begin{frame}
  \frametitle{Как проходят занятия}
  \begin{itemize}
  \item Обсуждаем
  \item Пробуем и тренеруемся
  \item Делимся опытом
  \item Между занятиями пишем пост
  \end{itemize}
\end{frame}

\section{Финальное задание}

\begin{frame}
  \frametitle{Финальное задание}
  Если коротко, то вы напишете свой пост, а преподаватель его проверит и даст обратную связь.
  \par Поэтапно:
  \begin{itemize}
  \item Определитесь с областью науки или технологии, которой будет посвящён пост.
  \item Определитесь со своей мотивацией и целью.
  \item Сформулируете тему.
  \item Составите план поста.
  \item Напишете пост.
  \item Перечитаете его и внесёте доработки и исправления.
  \item Дадите пост на вычитку друзьям и преподавателю.
  \item Получите обратную связь.
  \item Опубликуете пост (необязательно).
  \item Получите зачёт.
  \end{itemize}
\end{frame}

\section*{Благодарность}

\begin{frame}
  \center
  \textit{Благодарю за внимание!}
  
  \textbf{\textsl{\inserttitle}}

  \insertauthor

  \url{mailto:easimonenko@mail.ru}

  \insertinstitute
\end{frame}

\end{document}
